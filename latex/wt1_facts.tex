\documentclass[11pt]{amsart}
%%\documentclass{amsart}



\usepackage{amsmath,amsthm,amssymb}

\usepackage{hyperref}
\usepackage{cancel}


\usepackage{tikz}
\usetikzlibrary{matrix,arrows}



\theoremstyle{plain}

\newtheorem{thm}{Theorem}[section]
%\newtheorem{prop}{Proposition}[section]
\newtheorem{prop}[thm]{Proposition}
\newtheorem{lemma}[thm]{Lemma}
\newtheorem{cor}[thm]{Corollary}

\theoremstyle{definition}
\newtheorem{ex}[thm]{Example}
\newtheorem{defn}[thm]{Definition}
\newtheorem{remark}[thm]{Remark}
\newtheorem{conj}[thm]{Conjecture}




\newtheorem{lem}[thm]{Lemma}  
\newtheorem{rem}[thm]{Remark}  
\newtheorem{alg}[thm]{Algorithm}  
\newtheorem{prob}[thm]{Problem}  




%Letters
\newcommand{\A}{{\mathcal A}}
\newcommand{\bA}{{\bf A}}
\newcommand{\D}{{\mathcal D}}
\newcommand{\bD}{{\bf D}}
\newcommand{\E}{{\mathcal E}}
\newcommand{\F}{{\mathcal F}}
\newcommand{\K}{{\mathcal K}}
\renewcommand{\O}{{\mathcal O}}
\newcommand{\p}{{\mathfrak p}}
\newcommand{\M}{{\mathcal M}}
\newcommand{\U}{{\mathcal U}}
\renewcommand{\S}{{\mathcal S}}
\newcommand{\T}{{\mathbb T}}
\newcommand{\ve}{\varepsilon}
\newcommand{\oo}{\text{o}}
\newcommand{\OO}{\text{O}}
\renewcommand{\H}{\mathcal H}
\newcommand{\w}{\omega}

%Number systems
\newcommand{\N}{{\mathbb N}}
\newcommand{\Z}{{\mathbb Z}}
\newcommand{\Q}{{\mathbb Q}}
\newcommand{\R}{{\mathbb R}}
\newcommand{\C}{{\mathbb C}}
\newcommand{\Fp}{{\mathbb F}_p}
\newcommand{\Fpbar}{\overline{\mathbb F}_p}
\newcommand{\Qpbar}{\overline{{\Q}}_p}
\newcommand{\Qbar}{\overline{{\Q}}}
\newcommand{\Obar}{\overline{\O}}
\newcommand{\pbar}{\overline{{\mathfrak p}}}
\newcommand{\OK}{{\mathcal O}_K}
\renewcommand{\P}{{\mathbb P}}
\newcommand{\Zp}{\Z_p}
\newcommand{\Zpx}{\Z_p^\times}
\newcommand{\Qpx}{\Q_p^\times}
\newcommand{\Cpx}{\C_p^\times}
\newcommand{\Qp}{\Q_p}
\newcommand{\Cp}{\C_p}

%Maps
\newcommand{\inj}{\hookrightarrow}
\newcommand{\surj}{\twoheadrightarrow}
\newcommand{\maps}{\rightarrow}
\newcommand{\lra}{\longrightarrow}

%Matrices
\newcommand{\mat}{\begin{pmatrix} a & b \\ c & d \end{pmatrix}}
\newcommand{\smallmat}{\bigl( \begin{smallmatrix} a & b \\ c & d \end{smallmatrix} \bigr)}
\newcommand{\sigop}{\Sigma_0(p)}
\newcommand{\sop}{S_0(p)}
\newcommand{\psmallmat}[4]{\left( \begin{smallmatrix} #1 & #2 \\ #3 & #4 \end{smallmatrix} \right)}
\newcommand{\pmat}[4]{\left( \begin{matrix} #1 & #2 \\ #3 & #4 \end{matrix} \right)}
\newcommand{\bap}{\beta(a,p)}
\newcommand{\bapn}{\beta(a,p^n)}


%Random
\newcommand{\wt}{\widetilde}
\newcommand{\ol}{\overline}
\renewcommand{\binom}[2]{\genfrac{(}{)}{0pt}{}{#1}{#2}}
\newcommand{\Dinf}{\{0\} - \{\infty\}}
\renewcommand{\sp}{{\rho^*_k}}
\newcommand{\spd}{{\rho_k}}
\newcommand{\olsp}{{\ol{\rho}^*_k}}

%Spaces
\newcommand{\Arigv}[1]{\bA[#1]}
\newcommand{\Avrigv}[2]{\bA_{#1}[#2]}
\newcommand{\Drigv}[1]{\bD[#1]}
\newcommand{\Dvrigv}[2]{\bD_{#1}[#2]}
\newcommand{\Dvrig}[1]{\bD_{#1}}
\newcommand{\Drig}{\bD}
\newcommand{\Dkrig}{\Dvrig{k}}

\newcommand{\Avocv}[2]{\A_{#1}^\dag(\Zp,#2)}
\newcommand{\Avoc}[1]{\A_{#1}^\dag(\Zp,1)}
\newcommand{\Aocv}[1]{\A^\dag(\Zp,#1)}
\newcommand{\Akoc}{\Avoc{k}}
\newcommand{\Aoc}{\Aocv{1}}
\newcommand{\Dvocv}[1]{\D_{#1}^\dag(\Zp,#1)}
\newcommand{\Dvoc}[1]{\D_{#1}^\dag(\Zp,1)}
\newcommand{\Docv}[1]{\D^\dag(\Zp,#1)}
\newcommand{\Dkocv}[1]{\D^\dag(\Zp,#1)}
\newcommand{\Dkoc}{\D_k^\dag}
\newcommand{\Doc}{\D^\dag}


\newcommand{\Ala}{\A(\Zp)}
\newcommand{\Avla}[1]{\A_{#1}(\Zp)}
\newcommand{\Akla}{\Avla{k}}
\newcommand{\Dvla}[1]{\D_{#1}(\Zp)}
\newcommand{\Dla}{\D(\Zp)}
\newcommand{\Dkla}{\Dvla{k}}
\newcommand{\Dk}{\D_{k}}

\newcommand{\FkM}{\F_k(M)}



%Math Operators
\DeclareMathOperator{\Div}{Div} 
\DeclareMathOperator{\Hom}{Hom}
\DeclareMathOperator{\ord}{ord} 
\DeclareMathOperator{\PGL}{PGL}
\DeclareMathOperator{\GL}{GL} 
\DeclareMathOperator{\SL}{SL}
\DeclareMathOperator{\PSL}{PSL} 
\DeclareMathOperator{\Symb}{Symb}
\DeclareMathOperator{\Sym}{Sym} 
\DeclareMathOperator{\coker}{coker} 
\DeclareMathOperator{\im}{im}
\DeclareMathOperator{\Fil}{Fil}
\DeclareMathOperator{\trunc}{trunc}
\DeclareMathOperator{\nil}{nil}
\DeclareMathOperator{\ordin}{ord}
\DeclareMathOperator{\new}{new}
\DeclareMathOperator{\tr}{trace}


%Modular Symbols
\newcommand{\MS}[1]{\Symb_{\Gamma}(#1)}
\newcommand{\MSo}[1]{\Symb_{\Gamma_0}(#1)}
\newcommand{\Pfl}{\Phi_{f,\lambda}}

\newcommand{\rhobar}{\overline{\rho}}





%% Jon's Commands

\newcommand{\GCD}{\mathrm{GCD}}
\newcommand{\Char}{\mathrm{Char}}
\newcommand{\Rad}{\mathrm{Rad}}

\newcommand{\Ker}{\mathrm{Ker}}
\renewcommand{\Im}{\mathrm{Im}}

\newcommand{\NormF}{\mathrm{N}_{F/\Q}}



\newcommand{\even}{\text{-even}}
\newcommand{\aevenatp}{\a\even\text{ at $\p$}}

\newcommand{\B}{\mathcal{B}}


\newcommand{\x}{\vec{x}}
\newcommand{\y}{\vec{y}}
\renewcommand{\v}{\vec{v}}

\newcommand{\vr}{\vec{r}}
\newcommand{\vl}{\vec{l}}
\newcommand{\va}{\vec{a}}
\newcommand{\vi}{\vec{\iota}}



%\newcommand{\F}{\mathbb{F}}
\renewcommand{\P}{\mathbb{P}}

\renewcommand{\SS}{\mathbb{S}}
\newcommand{\DD}{\mathbb{D}}


\renewcommand{\O}{\mathcal{O}}



\renewcommand{\a}{\mathfrak{a}}
\renewcommand{\b}{\mathfrak{b}}
\renewcommand{\c}{\mathfrak{c}}

\renewcommand{\l}{\mathfrak{l}}
\newcommand{\n}{\mathfrak{n}}
\newcommand{\s}{\mathfrak{s}}



\newcommand{\al}{\alpha}



\newcommand{\ra}{\rightarrow}





\newcommand{\half}{\tfrac{1}{2}}



\renewcommand{\L}{\errorrrrrr}



\newcommand{\latI}{\mathcal{I}}
\newcommand{\latK}{\mathcal{K}}
\newcommand{\latL}{\mathcal{L}}
\newcommand{\latM}{\mathcal{M}}
\newcommand{\latN}{\mathcal{N}}



\newcommand{\Span}{\text{Span}}


\newcommand{\pip}{\pi_\p}
\newcommand{\q}{{\mathfrak q}}


\newcommand{\leg}[2]{\left( \frac{#1}{#2} \right)}

\DeclareMathOperator{\Tr}{trace}


\newcommand{\charpoly}{\text{CharPoly}}
\newcommand{\Frob}{\text{Frob}}
\renewcommand{\[}{\left[}
\renewcommand{\]}{\right]}

\newcommand{\Tcal}{\mathcal{T}}
\newcommand{\End}{\text{End}}
\newcommand{\m}{\mathfrak{m}}

\newcommand{\dbar}{\overline{d}}


\begin{document}

\subsection{Steinberg fact}
Consider $\chi$ a Dirichlet character of modulus $N$ and conductor $N_\chi$.  Write $N = N_\chi \cdot N'$.  Let $d_\chi$ denote the order of $\chi$.

~\\

\noindent
{\bf Fact:} if $q \nmid N_\chi$ and $q || N'$, then $S_1^{\new}(N,\chi) = 0$.  

~\\

The reason is easy.  At such a $q$, the associated Galois representation is Steinberg and thus has infinite image.  But this is not possible for a weight 1 form.

Here' something new.  If $\ord_
\ell(N) = \ord_\ell(N_\chi)$, then the Galois representation at $\ell$ has an unramified quotient which sends $\Frob_\ell$ to $a_\ell$.  In particular, for $\sigma \in G_{\Q_\ell}$ lifting $\Frob_\ell$, we have
$$
\rho(\sigma) = 
\begin{pmatrix}
\chi(\sigma) a_\ell^{-1} & 0 \\
0 & a_\ell
\end{pmatrix}.
$$
If $\overline{d}_\sigma$ is the projective order of $\rho(\sigma)$, then we have 
$$
\chi(\sigma)^{\overline{d}_\sigma} a_\ell^{-\overline{d}_\sigma} = a_\ell^{\overline{d}_\sigma}.
$$
Write $\chi = \chi_\ell \cdot \chi^{\ell}$ which $\chi_\ell$ is a character of modulus a power of $\ell$ and $\chi^{\ell}$ has modulus prime to $\ell$.  Then 
$$
\chi_\ell(\sigma)^{\overline{d}_\sigma} \chi^\ell(\ell)^{\overline{d}_\sigma}  a_\ell^{-\overline{d}_\sigma} = a_\ell^{\overline{d}_\sigma}
$$
nothing that $\chi^\ell(\sigma) = \chi^\ell(\ell)$ as $\sigma$ lifts $\Frob_\ell$.

Choosing $\sigma$ such that $\sigma_\ell(\sigma) = 1$ then gives
$$
a_\ell^{2 \overline{d}} = \chi^{\ell}(\ell)^{2\overline{d}}
$$
where $\overline{d}$ is in $\{1,2,3,4,5\}$.

On the other hand, if $D$ is the gcd of two elements of $\{1,2,3,4,5\}$, then 
$$
\chi_\ell(\sigma)^D \chi^\ell(\ell)^D  a_\ell^{-D} = a_\ell^D
$$
and
$$
\chi^\ell(\ell)^D  a_\ell^{-D} = a_\ell^D.
$$
In particular,
$$
\chi_\ell(\sigma)^D = 1
$$
for all $\sigma$.  Thus the order of $\chi_\ell$ divides $D$.  This puts a new condition on when weight 1 forms can exist!


\subsection{Fourier coefficients}

Let $f = \sum_n a_n q^n$ with $a_1=1$ be a newform in $S_1(N,\chi)$.\\

\begin{prop}~\\
\begin{enumerate}
\item if $\ell \nmid N$, then 
$$
a_\ell^2 = c \chi(\ell)
$$
where $c=0,1,2$ or $4$.

\item if $\ell | N_\chi$, $\ell \nmid N'$, then
$$
a_\ell^{2e} = 1
$$ 
where $e = d_\chi \overline{d} / \gcd(d_\chi,\overline{d})$ and $\overline{d}$ is a projective order:\ that is, $\overline{d} = 1,2,3,4$ or $5$.

\item if $\ell | N'$, then
$$
a_\ell = 0.
$$
\end{enumerate}
\end{prop}

\begin{proof}
The first part is Buzzard-Lauder, Lemma 1(b).

For the second part, if $\ell | N_\chi$ but $\ell \nmid N'$, then $\pi_\ell(f)$ is the ramified principal series $\pi(\chi_1,\chi_2)$ where $\chi_2$ is unramified with $\chi_2(\Frob_\ell) = a_\ell$ and $\chi_1 \chi_2 = \chi$ (Loeffler-Weinstein, Prop 2.8).  In particular, $\rho_f$ at $\ell$ is simply the direct sum $\chi_1 \oplus \chi_2$ (noting that the representation must be semi-simple as it is finite order).  Thus if $\sigma \in \Frob_\ell I_\ell$ with $I_\ell$ equal to inertia at $\ell$, then
$$
\rho_f(\sigma) = 
\begin{pmatrix}
\chi(\sigma)/a_{\ell} & 0 \\
0 & a_\ell \\
\end{pmatrix}
$$
and
$$
\rho_f(\sigma)^e = 
\begin{pmatrix}
a_{\ell}^{-e} & 0 \\
0 & a_\ell^e \\
\end{pmatrix}
$$
is a diagonal matrix as both $d_\chi$ and $\overline{d}$ divide $e$.  Thus, $a_\ell^{e} = a_{\ell}^{-e}$ and $a_{\ell}^{2e} =1$ as desired.

For the third part, $\pi_\ell(f)$ is supercuspidal again by the same Loeffler-Weinstein reference above which implies $a_\ell=0$.
\end{proof}

Let $\pi_{\ell}(x)$ denote the minimum polynomial of $a_\ell$ over $\Q$ and set $d_\ell$ equal to the degree of this polynomial.  Let $d_\chi = [\Q(\chi) : \Q]$ which is the order of $\chi$.

\begin{prop}
We have
$$
d_\ell \leq \gcd(d_\ell,d_\chi) \cdot \dim S_1(N,\chi).
$$
\end{prop}

\begin{proof}
If $K_f$ is the field of Fourier coefficients of $f$, then $[K_f:\Q(\chi)] \leq \dim S_1(N,\chi)$ as all of the $\Q(\chi)$-Galois conjugates of $f$ are in this weight 1 space.  Thus
\begin{align*}
S_1(N,\chi) 
&\geq [K_f:\Q(\chi)] \\
&\geq [\Q(\chi,a_\ell):\Q(\chi)] \\
&= [\Q(a_\ell):\Q(\chi) \cap \Q(a_\ell)] \\
&= [\Q(a_\ell):\Q] / [\Q(\chi) \cap \Q(a_\ell):\Q] \\
&\geq d_\ell / \gcd(d_\ell, d_\chi).
\end{align*}
Here the first equalities follows since $K_f/\Q$ is an abelian extension and last follows since $[\Q(\chi) \cap \Q(a_\ell):\Q]$ divides both $d_\chi$ and $d_\ell$.
\end{proof}
\vfill
\pagebreak
Let $\rho : G_{\Q} \to \GL_2(\C)$ be an Artin representation with conductor $N$ and determinant $\chi$.  In this note, we will describe all possible values of the trace of Frobenius at a prime $\ell \nmid N$.  

\begin{thm}
For $\ell \nmid N$, let $A = \rho(\Frob_\ell) \in \GL_2(\C)$ and let $\overline{A}$ denote the image of $A$ in $\PGL_2(\C)$.  If the order of $A$ is $d$, the order of $\overline{A}$ is $\overline{d}$, and the order of $\chi(\ell) \in \C^\times$ is $r$, then 
$$
d = \begin{cases}
\frac{2 r \dbar}{\gcd(r,\dbar)} & \text{~if~} 2 \mid \frac{r}{\gcd(r,\dbar)} \\ 
~\\
\frac{r \dbar}{\gcd(r,\dbar)} \text{~or~} \frac{2 r \dbar}{\gcd(r,\dbar)} & \text{~otherwise}
\end{cases}
$$
\end{thm}

\begin{proof}
Let $\langle A \rangle$ denote the cyclic subgroup of $\GL_2(\C)$ generated by $A$ and similarly for $\overline{A}$.  
We then have an exact sequence
$$
1 \to \langle A \rangle \cap \C^\times \to \langle A \rangle \to \langle \overline{A} \rangle \to 1.
$$
Since $\dbar$ is the smallest positive integer such that $A^{\dbar}$ is diagonal, this sequence is the same as 
$$
1 \to \langle A^{\dbar} \rangle  \to \langle A \rangle \to \langle \overline{A} \rangle \to 1.
$$
Thus, we have
$$
d = \dbar \cdot \ord(A^{\dbar}).
$$


To prove this theorem, we must then compute $\ord(A^{\dbar})$.  To this end, since $A^{\dbar}$ is diagonal with determinant $\chi(\ell)^{\dbar}$, we have
$$
A^{\dbar} = 
\begin{pmatrix}   \sqrt{\chi(\ell)^{\dbar}} & 0 \\ 0 & \sqrt{\chi(\ell)^{\dbar}} \end{pmatrix}
\text{~or~}
\begin{pmatrix}   -\sqrt{\chi(\ell)^{\dbar}} & 0 \\ 0 & -\sqrt{\chi(\ell)^{\dbar}} \end{pmatrix}
$$
where $\sqrt{\chi(\ell)^{\dbar}}$ is some fixed square root of ${\chi(\ell)^{\dbar}}$.
Thus, the order of $A^{\dbar}$ is simply the multiplicative order of $\pm  \sqrt{\chi(\ell)^{\dbar}}$.

We have $r = \ord(\chi(\ell))$.  Thus, $\ord(\sqrt{\chi(\ell)^{\dbar}}) = \displaystyle \frac{2r}{\gcd(r,\dbar)}$.  Now if $\zeta$ is a primitive $4n$-th root of unity, then the same is true of $-\zeta$.  However, if $\zeta$ is a primitive $2n$-th root of unity with $n$ odd, then $-\zeta$ is either a primitive $2n$-th root of unity or a primitive $n$-root of unity.  The theorem follows from this.
\end{proof}


\begin{remark}
We note that one can easily use this theorem to compute the trace of Frobenius as one can readily compute the trace of matrix given its order and determinant.  Indeed, if $A$ has order $d$ and determinant $m$, then it's eigenvalues are $\zeta_d$ and $\zeta_d^{-1} m$ where $\zeta_d$ is a primitive $d$-th root of unity and thus the trace of $A$ is $\zeta_d + m \zeta_d^{-1}$.  (NO!  THIS IS WRONG BUT EASILY FIXED -- NEED TO DEAL WITH DIVISORS OF $d$.)

Here's a table of traces of matrices with determinant 1:

\begin{center}
\begin{tabular}{|c|c|c|}
\hline
$d $  & $\zeta_d + \zeta_d^{-1}$ & trace \\
\hline
1 & 1 + 1 & 2 \\
2 & -1 + -1 & -2  \\
3 & $\zeta_3 + \zeta_3^{-1}$ & -1  \\
4 & $\zeta_4 + \zeta_4^{-1}$ & 0 \\
5 & $\zeta_5 + \zeta_5^{-1}$  & $\frac{-1 \pm \sqrt{5}}{2}$ \\
6 & $\zeta_6 + \zeta_6^{-1}$  & 1 \\
8 & $\zeta_8 + \zeta_8^{-1}$  & $\pm \sqrt{2}$ \\
10 & $\zeta_{10} + \zeta_{10}^{-1}$  & $\frac{1 \pm \sqrt{5}}{2}$ \\
\hline
\end{tabular}
\end{center}

Here's a table of traces of matrices with determinant -1:

\begin{center}
\begin{tabular}{|c|c|c|}
\hline
$d $  & $\zeta_d - \zeta_d^{-1}$ & trace \\
\hline
2 & -1 - -1 & 0  \\
%3 & $\zeta_3 - \zeta_3^{-1}$ & \pm \sqrt{-3}  \\
4 & $\zeta_4 - \zeta_4^{-1}$ &  $\pm 2 i$ \\
%5 & $\zeta_5 - \zeta_5^{-1}$  & $\frac{-1 \pm \sqrt{5}}{2}$ \\
%6 & $\zeta_6 + \zeta_6^{-1}$  & -1 \\
8 & $\zeta_8 - \zeta_8^{-1}$  & $\pm \sqrt{-2}$ \\
12 & $\zeta_{12} - \zeta_{12}^{-1}$  & $\pm i$ \\
20 & $\zeta_{20} - \zeta_{20}^{-1}$  & $\pm \sqrt{\frac{-3 \pm \sqrt{5}}{2}}$ \\
\hline
\end{tabular}
\end{center}

\end{remark}



Now we use this theorem to compare to Serre's claims on page 263 at the bottom of his weight 1 article.  Namely, let $N = q$ a prime and take $\chi$ to be a quadratic character.  In this case, the projective image of $\rho(G_\Q)$ is either $S_4$ or $A_5$.  

Let's first analyze the $S_4$ case.  In this group the possible orders are 1,2,3, and 4.  (These are the possible values of $\dbar$ from the last theorem.)  We proceed in two cases: $\chi(\ell) = 1$ or $\chi(\ell) = -1$; that is, $r=1$ or $r=2$. When $r = 1$, we have $\frac{r \dbar}{\gcd(r,\dbar)} = 1$ is always odd and $\frac{r \dbar}{\gcd(r,\dbar)} = \dbar$.  Thus 
\begin{itemize}
\item $\dbar = 1 \implies d = 1 \text{~or~} 2 \implies \tr = 2 \text{~or~} -2$
\item $\dbar = 2 \implies d = 2 \text{~or~} 4 \implies \tr = -2 \text{~or~} 0$
\item $\dbar = 3 \implies d = 3 \text{~or~} 6 \implies \tr = -1 \text{~or~} 1$
\item $\dbar = 4 \implies d = 4 \text{~or~} 8 \implies \tr = 0 \text{~or~}\pm \sqrt{2}$
\end{itemize}
Thus, the possible traces are $0, \pm 1, \pm 2, \pm \sqrt{2}$.  This exactly matches Serre would predicts that the traces have squares equal to 0,1,2 and 4.

When $r = 2$, we have
\begin{itemize}
\item $\dbar = 1 \implies \frac{r}{\gcd(r,\dbar)} = 2 \implies d = 4 \implies \tr = \pm 2i$
\item $\dbar = 2 \implies \frac{r}{\gcd(r,\dbar)} = 1 \implies d = 2 \text{~or~} 4 \implies \tr = 0, \pm 2i$
\item $\dbar = 3 \implies \frac{r}{\gcd(r,\dbar)} = 2 \implies d = 12 \implies \tr = \pm i$
\item $\dbar = 4 \implies \frac{r}{\gcd(r,\dbar)} = 1 \implies d = 4 \text{~or~} 8 \implies \tr = \pm 2i , \pm \sqrt{-2}$
\end{itemize}
Thus, the possible traces are $0, \pm i, \pm 2i, \pm \sqrt{-2}$.  This exactly matches Serre would predicts that the traces have squares equal to 0,-1,-2 and -4.  Great!

Now on to $A_5$.  We then have $\dbar = 1,2,3,$ or 5. Again, taking $r=1$,
we have $\frac{r \dbar}{\gcd(r,\dbar)} = 1$ is always odd and $\frac{r \dbar}{\gcd(r,\dbar)} = \dbar$.  Thus
\begin{itemize}
\item $\dbar = 1 \implies d = 1 \text{~or~} 2 \implies \tr = 2 \text{~or~} -2$
\item $\dbar = 2 \implies d = 2 \text{~or~} 4 \implies \tr = -2 \text{~or~} 0$
\item $\dbar = 3 \implies d = 3 \text{~or~} 6 \implies \tr = -1 \text{~or~} 1$
\item $\dbar = 5 \implies d = 5 \text{~or~} 10 \implies \tr = \frac{-1 \pm \sqrt{5}}{2} \text{~or~} \frac{1 \pm \sqrt{5}}{2} $
\end{itemize}
Thus, the possible traces are $0, \pm 1, \pm 2, \frac{\pm 1 \pm \sqrt{5}}{2}$.  This exactly matches Serre would predicts that the traces have squares equal to 0,1,4 and $\frac{3 \pm \sqrt{5}}{2}$.

When $r = 2$, we have
\begin{itemize}
\item $\dbar = 1 \implies \frac{r}{\gcd(r,\dbar)} = 2 \implies d = 4 \implies \tr = \pm 2i$
\item $\dbar = 2 \implies \frac{r}{\gcd(r,\dbar)} = 1 \implies d = 2 \text{~or~} 4 \implies \tr = 0, \pm 2i$
\item $\dbar = 3 \implies \frac{r}{\gcd(r,\dbar)} = 2 \implies d = 12 \implies \tr = \pm i$
\item $\dbar = 5 \implies \frac{r}{\gcd(r,\dbar)} = 2 \implies d = 20 \implies \tr = \pm \sqrt{\frac{-3 \pm \sqrt{5}}{2}}$.
\end{itemize}
Thus, the possible traces are $0, \pm i, \pm 2i, \pm \sqrt{\frac{-3 \pm \sqrt{5}}{2}}$.  This exactly matches Serre would predicts that the traces have squares equal to 0,-1,-4, and $\frac{-3 \pm \sqrt{5}}{2}$.  Great!

\end{document}